\documentclass[11pt]{report}
% \usepackage[dissertation,final,raggedbottom]{USCthesis}
\usepackage[dissertation,final,copyright,raggedbottom]{USCthesis}

% \usepackage{fontspec}
% \setromanfont{Noto Sans}

%%% CITATION FORMATTING %%%

$if(csl-refs)$
% definitions for citeproc citations
\NewDocumentCommand\citeproctext{}{}
\NewDocumentCommand\citeproc{mm}{%
  \begingroup\def\citeproctext{#2}\cite{#1}\endgroup}
\makeatletter
 % allow citations to break across lines
 \let\@cite@ofmt\@firstofone
 % avoid brackets around text for \cite:
 \def\@biblabel#1{}
 \def\@cite#1#2{{#1\if@tempswa , #2\fi}}
\makeatother
\newlength{\cslhangindent}
\setlength{\cslhangindent}{1.5em}
\newlength{\csllabelwidth}
\setlength{\csllabelwidth}{3em}
\newenvironment{CSLReferences}[2] % #1 hanging-indent, #2 entry-spacing
 {\begin{list}{}{%
  \setlength{\itemindent}{0pt}
  \setlength{\leftmargin}{0pt}
  \setlength{\parsep}{0pt}
  % turn on hanging indent if param 1 is 1
  \ifodd #1
   \setlength{\leftmargin}{\cslhangindent}
   \setlength{\itemindent}{-1\cslhangindent}
   \setlength{\parindent}{0pt}
  \fi
  % set entry spacing
  \setlength{\itemsep}{#2\baselineskip}}}
 {\end{list}}
\usepackage{calc}
\newcommand{\CSLBlock}[1]{\hfill\break#1\hfill\break}
\newcommand{\CSLLeftMargin}[1]{\parbox[t]{\csllabelwidth}{\strut#1\strut}}
\newcommand{\CSLRightInline}[1]{\parbox[t]{\linewidth - \csllabelwidth}{\strut#1\strut}}
\newcommand{\CSLIndent}[1]{\hspace{\cslhangindent}#1}
$endif$
\setlength{\parindent}{50pt}
\def\tightlist{}

% %%% JUNK %%%
% %% inserted to fix a CSL error
% \newlength{\cslhangindent}
% \setlength{\cslhangindent}{1.5em}
% \newenvironment{CSLReferences}%
% {\setlength{\parindent}{1pt}%
% \everypar{\setlength{\hangindent}{\cslhangindent}}\ignorespaces}%
% {\par}
% \def\tightlist{}

% guidelines for manuscript formatting: https://graduateschool.usc.edu/wp-content/uploads/2020/11/Manuscript_Formatting_and_Documentation_Styles.pdf
%,tocbold        % uncomment to enable bold chapter titles in the ToC
                 %
                 % the style guidelines state that page numbers in the
                 % ToC should not be bold, but leave it up to the author
                 % and specific department guidelines as to how the
                 % chapter (or other section) titles should be typeset.
%% our customizations %%%%%%%%%%%%%%%%%%%%%%%%%%%%%%%%%%%%%%%%%%%%%%%%%%
% \usepackage[export]{adjustbox} % for frame option in \includegraphics
% \usepackage{amsmath}
% \usepackage{amssymb}
% \usepackage{array}
\usepackage[utf8]{inputenc} % load inputenc before csquotes
% \usepackage[english]{babel}
% \usepackage[
%   backend     = biber,
%   doi         = true,
%   hyperref    = true,
%   maxbibnames = 99,
%   sortlocale  = en_US,
%   style       = numeric,
% ]{biblatex}
% \usepackage{booktabs}
\usepackage{color, colortbl}
\usepackage{csquotes}
\usepackage{efbox}
\usepackage{enumitem}
\usepackage[shortcuts]{extdash} % use `\-/' to hyphenate words/phrases that have a dash in them
\usepackage[tt=false]{libertine} % libertine's \ttfamily isn't that great
% \usepackage[T1]{fontenc} % load fonts before fontenc
% \usepackage[perpage]{bigfoot}
\usepackage[
  showframe = false,% draw a border around textwidth
  pass      = true, % force 8.5"x11" pagesize
]{geometry}
\usepackage{graphicx}
\usepackage[notquote]{hanging} % enables negative indents in paragraphs
\usepackage{hyphenat}
\usepackage{ifthen}
\usepackage{lipsum}
\usepackage{multirow}
\usepackage{parnotes}
% \usepackage{pdflscape} % rotate some pages in an {landscape} environment
\usepackage{pifont}
\usepackage{ragged2e}
\usepackage{seqsplit}
% \usepackage{siunitx}
\usepackage{subcaption}
\usepackage{tabularx}
\usepackage{xcolor}
\usepackage{xspace}
\usepackage{url}

\usepackage[
  breaklinks    = true,
  colorlinks    = true,
  hypertexnames = false,
  pdfpagelabels = false,
  citecolor     = {blue!80!black},
  linkcolor     = {blue!80!black},
  urlcolor      = {blue!80!black},
]{hyperref} % load hyperref as the last package

% pkg: biblatex
% \setlength\bibitemsep{0.5\baselineskip}                 % add a line between entries
% \AtEveryBibitem{\iffieldundef{doi}{}{\clearfield{url}}} % if DOI, hide URL

% % \addbibresource{includes/bibliography.json}
% \addbibresource{dissbib.bib}

% % pkg: siunitx
% % some guidelines https://physics.nist.gov/cuu/Units/checklist.html
% \sisetup{
%   tight-spacing  = true
%   ,detect-family = true
%   ,detect-mode   = true
%   ,binary-units  = true    % support for MB, GB, etc.
%   ,range-units   = single  % "3% to 5%" -> "3 to 5%"
%   ,range-phrase  = --      % "3 to 5%"  -> "3--5%"
% }

% pkg: babel, hyperref
% \addto\extrasenglish{%
%   \renewcommand{\chapterautorefname}{Chapter}
%   \renewcommand{\sectionautorefname}{Section}
%   \renewcommand{\subsectionautorefname}{Section}
%   \renewcommand{\subsubsectionautorefname}{Section}
% }

% pkg: url
\renewcommand{\UrlFont}{\footnotesize\tt}
% \renewcommand{\thefootnote}{\arabic{footnote}}

% our custom commands
\renewcommand{\ttdefault}{cmtt} % use computer modern for teletype


%%% draft mode / toggle commands %%%
\usepackage{etoolbox}
\newtoggle{draft}
\settoggle{draft}{false} % change toggle for draft or final versions

\iftoggle{draft}{
  % if 'draft' toggle is true
  \overfullrule=10pt                       % highlight overfull hboxes
}{
  % if 'draft' toggle is false
  \PassOptionsToPackage{final}{showlabels} % hide labels on figures, etc
}

% if you're including existing papers into your thesis, it helps to put
% content behind a toggle (or conditional) so you only have to maintain
% and keep consistency on one copy. see "introduction.tex".
\newtoggle{thesis}
\settoggle{thesis}{true}

\usepackage[inline]{showlabels}
\renewcommand{\showlabelfont}{\sffamily \color{blue}}
\renewcommand{\showlabelsetlabel}[1]{\efbox{\showlabelfont #1}}
%%%%%%%%%%%%%%%%%%%%%%%%%%%%%%%%%%%%%%%%%%%%%%%%%%%%%%%%%%%%%%%%%%%%%%%%

%%% front matter %%%%%%%%%%%%%%%%%%%%%%%%%%%%%%%%%%%%%%%%%%%%%%%%%%%%%%%
\begin{document}
\title{CODING.CARE: GUIDEBOOKS FOR INTERSECTIONAL AI}
% \subtitle{GUIDEBOOKS FOR INTERSECTIONAL AI}
\author{Sarah Ciston}
\majorfield{CINEMATIC ARTS (MEDIA ARTS AND PRACTICE)}
\submitdate{May 2024}

\setcounter{tocdepth}{1} 

%%% preface %%%%%%%%%%%%%%%%%%%%%%%%%%%%%%%%%%%%%%%%%%%%%%%%%%%%%%%%%%%%
\begin{preface}
  % \prefacesection{Dedication}
  % \input{dedication.tex}

  \prefacesection{Acknowledgements}
  % \input{acknowledgements.tex}

  {
  \hypersetup{hidelinks} % color all links black in the preface
  \tableofcontents
  % \listoftables
  \listoffigures
  }

  \prefacesection{Abstract}
  % \input{abstract.tex}
\end{preface}
\raggedright

%%% introduction %%%%%%%%%%%%%%%%%%%%%%%%%%%%%%%%%%%%%%%%%%%%%%%%%%%%%%%
\chapter{Crafting Trans*formative Systems}
  % \label{ch:introduction}

\graphicspath{}
% \input{introduction.tex}

$body$

\addcontentsline{toc}{chapter}{*: (Un)Raveling and (Un)Limiting}

\addcontentsline{toc}{chapter}{FORMATIVE: "A Critical Field Guide for Working with Machine Learning Datasets" and "Inclusive Datasets Research Guide"}

% \addcontentsline{toc}{chapter}{TECH: Intersectional AI Toolkit}
\chapter{TECH: Intersectional AI Toolkit}

To explore the Intersectional AI Toolkit, visit \url:{https://intersectionalai.com}

\addcontentsline{toc}{chapter}{NO: (Un)Forming, (Un)Living, and (Un)Knowing}

% \addcontentsline{toc}{chapter}{CRAFT: Coding.Care: Field Notes for Making Friends with Code}

\chapter{CRAFT: Coding.Care: Field Notes for Making Friends with Code}

To read Coding.Care: Field Notes for Making Friends with Code, visit \url:{https://coding.care}, where you can also explore the entire collection of Trans*formative TechnoCraft materials.

%%% chapter %%%%%%%%%%%%%%%%%%%%%%%%%%%%%%%%%%%%%%%%%%%%%%%%%%%%%%%%%%%%

% if you're "stapling" together papers, it's easy to include your paper
% directory by way of symlinks, or copying the entire paper as a
% subdirectory.
%
% for example, if your paper directory looks like the following:
%
%   foobar/          - top level paper directory
%   foobar/fig/      - where all graphics and figures live
%   foobar/paper.bib - bibliography
%   foobar/paper.tex - monolithic .tex file for paper
%
% then you might use the folloiwng:
%
%   \graphicspath{foobar/fig}
%   \addbibresource{foobar/paper.bib}
%   \documentclass[11pt]{report}
\usepackage[
  dissertation
 ,final
 ,raggedbottom
%,tocbold        % uncomment to enable bold chapter titles in the ToC
                 %
                 % the style guidelines state that page numbers in the
                 % ToC should not be bold, but leave it up to the author
                 % and specific department guidelines as to how the
                 % chapter (or other section) titles should be typeset.
]{USCthesis}

% guidelines for manuscript formatting: https://graduateschool.usc.edu/wp-content/uploads/2020/11/Manuscript_Formatting_and_Documentation_Styles.pdf

%% our customizations %%%%%%%%%%%%%%%%%%%%%%%%%%%%%%%%%%%%%%%%%%%%%%%%%%
\usepackage[export]{adjustbox} % for frame option in \includegraphics
\usepackage{amsmath}
\usepackage{amssymb}
\usepackage{array}
\usepackage[utf8]{inputenc} % load inputenc before csquotes
\usepackage[english]{babel}
\usepackage[
  backend     = biber,
  doi         = true,
  hyperref    = true,
  maxbibnames = 99,
  sortlocale  = en_US,
  style       = numeric,
]{biblatex}
\usepackage{booktabs}
\usepackage{color, colortbl}
\usepackage{csquotes}
\usepackage{efbox}
\usepackage{enumitem}
\usepackage[shortcuts]{extdash} % use `\-/' to hyphenate words/phrases that have a dash in them
\usepackage[tt=false]{libertine} % libertine's \ttfamily isn't that great
% \usepackage[T1]{fontenc} % load fonts before fontenc
\usepackage[symbol]{footmisc}
\usepackage[
  showframe = false,% draw a border around textwidth
  pass      = true, % force 8.5"x11" pagesize
]{geometry}
\usepackage{graphicx}
%\usepackage[notquote]{hanging} % enables negative indents in paragraphs
\usepackage{hyphenat}
\usepackage{ifthen}
\usepackage{lipsum}
\usepackage{multirow}
\usepackage{parnotes}
\usepackage{pdflscape} % rotate some pages in an {landscape} environment
\usepackage{pifont}
\usepackage{ragged2e}
\usepackage{seqsplit}
\usepackage{siunitx}
\usepackage{subcaption}
\usepackage{tabularx}
\usepackage{xcolor}
\usepackage{xspace}
\usepackage{url}

\usepackage[
  breaklinks    = true,
  colorlinks    = true,
  hypertexnames = false,
  pdfpagelabels = false,
  citecolor     = {blue!80!black},
  linkcolor     = {blue!80!black},
  urlcolor      = {blue!80!black},
]{hyperref} % load hyperref as the last package

% pkg: biblatex
\setlength\bibitemsep{0.5\baselineskip}                 % add a line between entries
\AtEveryBibitem{\iffieldundef{doi}{}{\clearfield{url}}} % if DOI, hide URL

% \addbibresource{paper.bib}
\addbibresource{includes/bibliography.json}
\def\tightlist{}
% pkg: siunitx
% some guidelines https://physics.nist.gov/cuu/Units/checklist.html
\sisetup{
  tight-spacing  = true
  ,detect-family = true
  ,detect-mode   = true
  ,binary-units  = true    % support for MB, GB, etc.
  ,range-units   = single  % "3% to 5%" -> "3 to 5%"
  ,range-phrase  = --      % "3 to 5%"  -> "3--5%"
}

% pkg: babel, hyperref
\addto\extrasenglish{%
  \renewcommand{\chapterautorefname}{Chapter}
  \renewcommand{\sectionautorefname}{Section}
  \renewcommand{\subsectionautorefname}{Section}
  \renewcommand{\subsubsectionautorefname}{Section}
}

% pkg: url
\renewcommand{\UrlFont}{\footnotesize\tt}
% \renewcommand{\thefootnote}{\arabic{footnote}}

% our custom commands
\renewcommand{\ttdefault}{cmtt} % use computer modern for teletype

%%% draft mode / toggle commands %%%
\usepackage{etoolbox}
\newtoggle{draft}
\settoggle{draft}{true} % change toggle for draft or final versions

\iftoggle{draft}{
  % if 'draft' toggle is true
  \overfullrule=10pt                       % highlight overfull hboxes
}{
  % if 'draft' toggle is false
  \PassOptionsToPackage{final}{showlabels} % hide labels on figures, etc
}

% if you're including existing papers into your thesis, it helps to put
% content behind a toggle (or conditional) so you only have to maintain
% and keep consistency on one copy. see "introduction.tex".
\newtoggle{thesis}
\settoggle{thesis}{true}

\usepackage[inline]{showlabels}
\renewcommand{\showlabelfont}{\sffamily \color{blue}}
\renewcommand{\showlabelsetlabel}[1]{\efbox{\showlabelfont #1}}
%%%%%%%%%%%%%%%%%%%%%%%%%%%%%%%%%%%%%%%%%%%%%%%%%%%%%%%%%%%%%%%%%%%%%%%%

%%% front matter %%%%%%%%%%%%%%%%%%%%%%%%%%%%%%%%%%%%%%%%%%%%%%%%%%%%%%%
\begin{document}

% title should be all caps
\title{CODING.CARE: GUIDEBOOKS FOR INTERSECTIONAL AI}
% use your full name!
% https://cs.stanford.edu/~knuth/news19.html
% "Let's celebrate everybody's full names"
\author{Sarah Ciston}

% major should be all caps
\majorfield{CINEMATIC ARTS (MEDIA ARTS AND PRACTICE)}

% date should be May, August, or December (when degrees are conferred)
\submitdate{May 2024}

%%% preface %%%%%%%%%%%%%%%%%%%%%%%%%%%%%%%%%%%%%%%%%%%%%%%%%%%%%%%%%%%%
\begin{preface}
  % \prefacesection{Dedication}
  % \input{dedication.tex}

  \prefacesection{Acknowledgements}
  % \input{acknowledgements.tex}

  {
  \hypersetup{hidelinks} % color all links black in the preface
  \tableofcontents
  \listoftables
  \listoffigures
  }

  \prefacesection{Abstract}
  % \input{abstract.tex}
\end{preface}

%%% introduction %%%%%%%%%%%%%%%%%%%%%%%%%%%%%%%%%%%%%%%%%%%%%%%%%%%%%%%
\chapter{Introduction}
  \label{ch:introduction}

\graphicspath{}
% \input{introduction.tex}
$body$


%%% chapter %%%%%%%%%%%%%%%%%%%%%%%%%%%%%%%%%%%%%%%%%%%%%%%%%%%%%%%%%%%%

% if you're "stapling" together papers, it's easy to include your paper
% directory by way of symlinks, or copying the entire paper as a
% subdirectory.
%
% for example, if your paper directory looks like the following:
%
%   foobar/          - top level paper directory
%   foobar/fig/      - where all graphics and figures live
%   foobar/paper.bib - bibliography
%   foobar/paper.tex - monolithic .tex file for paper
%
% then you might use the folloiwng:
%
%   \graphicspath{foobar/fig}
%   \addbibresource{foobar/paper.bib}
%   \documentclass[11pt]{report}
\usepackage[
  dissertation
 ,final
 ,raggedbottom
%,tocbold        % uncomment to enable bold chapter titles in the ToC
                 %
                 % the style guidelines state that page numbers in the
                 % ToC should not be bold, but leave it up to the author
                 % and specific department guidelines as to how the
                 % chapter (or other section) titles should be typeset.
]{USCthesis}

% guidelines for manuscript formatting: https://graduateschool.usc.edu/wp-content/uploads/2020/11/Manuscript_Formatting_and_Documentation_Styles.pdf

%% our customizations %%%%%%%%%%%%%%%%%%%%%%%%%%%%%%%%%%%%%%%%%%%%%%%%%%
\usepackage[export]{adjustbox} % for frame option in \includegraphics
\usepackage{amsmath}
\usepackage{amssymb}
\usepackage{array}
\usepackage[utf8]{inputenc} % load inputenc before csquotes
\usepackage[english]{babel}
\usepackage[
  backend     = biber,
  doi         = true,
  hyperref    = true,
  maxbibnames = 99,
  sortlocale  = en_US,
  style       = numeric,
]{biblatex}
\usepackage{booktabs}
\usepackage{color, colortbl}
\usepackage{csquotes}
\usepackage{efbox}
\usepackage{enumitem}
\usepackage[shortcuts]{extdash} % use `\-/' to hyphenate words/phrases that have a dash in them
\usepackage[tt=false]{libertine} % libertine's \ttfamily isn't that great
% \usepackage[T1]{fontenc} % load fonts before fontenc
\usepackage[symbol]{footmisc}
\usepackage[
  showframe = false,% draw a border around textwidth
  pass      = true, % force 8.5"x11" pagesize
]{geometry}
\usepackage{graphicx}
%\usepackage[notquote]{hanging} % enables negative indents in paragraphs
\usepackage{hyphenat}
\usepackage{ifthen}
\usepackage{lipsum}
\usepackage{multirow}
\usepackage{parnotes}
\usepackage{pdflscape} % rotate some pages in an {landscape} environment
\usepackage{pifont}
\usepackage{ragged2e}
\usepackage{seqsplit}
\usepackage{siunitx}
\usepackage{subcaption}
\usepackage{tabularx}
\usepackage{xcolor}
\usepackage{xspace}
\usepackage{url}

\usepackage[
  breaklinks    = true,
  colorlinks    = true,
  hypertexnames = false,
  pdfpagelabels = false,
  citecolor     = {blue!80!black},
  linkcolor     = {blue!80!black},
  urlcolor      = {blue!80!black},
]{hyperref} % load hyperref as the last package

% pkg: biblatex
\setlength\bibitemsep{0.5\baselineskip}                 % add a line between entries
\AtEveryBibitem{\iffieldundef{doi}{}{\clearfield{url}}} % if DOI, hide URL

% \addbibresource{paper.bib}
\addbibresource{includes/bibliography.json}
\def\tightlist{}
% pkg: siunitx
% some guidelines https://physics.nist.gov/cuu/Units/checklist.html
\sisetup{
  tight-spacing  = true
  ,detect-family = true
  ,detect-mode   = true
  ,binary-units  = true    % support for MB, GB, etc.
  ,range-units   = single  % "3% to 5%" -> "3 to 5%"
  ,range-phrase  = --      % "3 to 5%"  -> "3--5%"
}

% pkg: babel, hyperref
\addto\extrasenglish{%
  \renewcommand{\chapterautorefname}{Chapter}
  \renewcommand{\sectionautorefname}{Section}
  \renewcommand{\subsectionautorefname}{Section}
  \renewcommand{\subsubsectionautorefname}{Section}
}

% pkg: url
\renewcommand{\UrlFont}{\footnotesize\tt}
% \renewcommand{\thefootnote}{\arabic{footnote}}

% our custom commands
\renewcommand{\ttdefault}{cmtt} % use computer modern for teletype

%%% draft mode / toggle commands %%%
\usepackage{etoolbox}
\newtoggle{draft}
\settoggle{draft}{true} % change toggle for draft or final versions

\iftoggle{draft}{
  % if 'draft' toggle is true
  \overfullrule=10pt                       % highlight overfull hboxes
}{
  % if 'draft' toggle is false
  \PassOptionsToPackage{final}{showlabels} % hide labels on figures, etc
}

% if you're including existing papers into your thesis, it helps to put
% content behind a toggle (or conditional) so you only have to maintain
% and keep consistency on one copy. see "introduction.tex".
\newtoggle{thesis}
\settoggle{thesis}{true}

\usepackage[inline]{showlabels}
\renewcommand{\showlabelfont}{\sffamily \color{blue}}
\renewcommand{\showlabelsetlabel}[1]{\efbox{\showlabelfont #1}}
%%%%%%%%%%%%%%%%%%%%%%%%%%%%%%%%%%%%%%%%%%%%%%%%%%%%%%%%%%%%%%%%%%%%%%%%

%%% front matter %%%%%%%%%%%%%%%%%%%%%%%%%%%%%%%%%%%%%%%%%%%%%%%%%%%%%%%
\begin{document}

% title should be all caps
\title{CODING.CARE: GUIDEBOOKS FOR INTERSECTIONAL AI}
% use your full name!
% https://cs.stanford.edu/~knuth/news19.html
% "Let's celebrate everybody's full names"
\author{Sarah Ciston}

% major should be all caps
\majorfield{CINEMATIC ARTS (MEDIA ARTS AND PRACTICE)}

% date should be May, August, or December (when degrees are conferred)
\submitdate{May 2024}

%%% preface %%%%%%%%%%%%%%%%%%%%%%%%%%%%%%%%%%%%%%%%%%%%%%%%%%%%%%%%%%%%
\begin{preface}
  % \prefacesection{Dedication}
  % \input{dedication.tex}

  \prefacesection{Acknowledgements}
  % \input{acknowledgements.tex}

  {
  \hypersetup{hidelinks} % color all links black in the preface
  \tableofcontents
  \listoftables
  \listoffigures
  }

  \prefacesection{Abstract}
  % \input{abstract.tex}
\end{preface}

%%% introduction %%%%%%%%%%%%%%%%%%%%%%%%%%%%%%%%%%%%%%%%%%%%%%%%%%%%%%%
\chapter{Introduction}
  \label{ch:introduction}

\graphicspath{}
% \input{introduction.tex}
$body$


%%% chapter %%%%%%%%%%%%%%%%%%%%%%%%%%%%%%%%%%%%%%%%%%%%%%%%%%%%%%%%%%%%

% if you're "stapling" together papers, it's easy to include your paper
% directory by way of symlinks, or copying the entire paper as a
% subdirectory.
%
% for example, if your paper directory looks like the following:
%
%   foobar/          - top level paper directory
%   foobar/fig/      - where all graphics and figures live
%   foobar/paper.bib - bibliography
%   foobar/paper.tex - monolithic .tex file for paper
%
% then you might use the folloiwng:
%
%   \graphicspath{foobar/fig}
%   \addbibresource{foobar/paper.bib}
%   \documentclass[11pt]{report}
\usepackage[
  dissertation
 ,final
 ,raggedbottom
%,tocbold        % uncomment to enable bold chapter titles in the ToC
                 %
                 % the style guidelines state that page numbers in the
                 % ToC should not be bold, but leave it up to the author
                 % and specific department guidelines as to how the
                 % chapter (or other section) titles should be typeset.
]{USCthesis}

% guidelines for manuscript formatting: https://graduateschool.usc.edu/wp-content/uploads/2020/11/Manuscript_Formatting_and_Documentation_Styles.pdf

%% our customizations %%%%%%%%%%%%%%%%%%%%%%%%%%%%%%%%%%%%%%%%%%%%%%%%%%
\usepackage[export]{adjustbox} % for frame option in \includegraphics
\usepackage{amsmath}
\usepackage{amssymb}
\usepackage{array}
\usepackage[utf8]{inputenc} % load inputenc before csquotes
\usepackage[english]{babel}
\usepackage[
  backend     = biber,
  doi         = true,
  hyperref    = true,
  maxbibnames = 99,
  sortlocale  = en_US,
  style       = numeric,
]{biblatex}
\usepackage{booktabs}
\usepackage{color, colortbl}
\usepackage{csquotes}
\usepackage{efbox}
\usepackage{enumitem}
\usepackage[shortcuts]{extdash} % use `\-/' to hyphenate words/phrases that have a dash in them
\usepackage[tt=false]{libertine} % libertine's \ttfamily isn't that great
% \usepackage[T1]{fontenc} % load fonts before fontenc
\usepackage[symbol]{footmisc}
\usepackage[
  showframe = false,% draw a border around textwidth
  pass      = true, % force 8.5"x11" pagesize
]{geometry}
\usepackage{graphicx}
%\usepackage[notquote]{hanging} % enables negative indents in paragraphs
\usepackage{hyphenat}
\usepackage{ifthen}
\usepackage{lipsum}
\usepackage{multirow}
\usepackage{parnotes}
\usepackage{pdflscape} % rotate some pages in an {landscape} environment
\usepackage{pifont}
\usepackage{ragged2e}
\usepackage{seqsplit}
\usepackage{siunitx}
\usepackage{subcaption}
\usepackage{tabularx}
\usepackage{xcolor}
\usepackage{xspace}
\usepackage{url}

\usepackage[
  breaklinks    = true,
  colorlinks    = true,
  hypertexnames = false,
  pdfpagelabels = false,
  citecolor     = {blue!80!black},
  linkcolor     = {blue!80!black},
  urlcolor      = {blue!80!black},
]{hyperref} % load hyperref as the last package

% pkg: biblatex
\setlength\bibitemsep{0.5\baselineskip}                 % add a line between entries
\AtEveryBibitem{\iffieldundef{doi}{}{\clearfield{url}}} % if DOI, hide URL

% \addbibresource{paper.bib}
\addbibresource{includes/bibliography.json}
\def\tightlist{}
% pkg: siunitx
% some guidelines https://physics.nist.gov/cuu/Units/checklist.html
\sisetup{
  tight-spacing  = true
  ,detect-family = true
  ,detect-mode   = true
  ,binary-units  = true    % support for MB, GB, etc.
  ,range-units   = single  % "3% to 5%" -> "3 to 5%"
  ,range-phrase  = --      % "3 to 5%"  -> "3--5%"
}

% pkg: babel, hyperref
\addto\extrasenglish{%
  \renewcommand{\chapterautorefname}{Chapter}
  \renewcommand{\sectionautorefname}{Section}
  \renewcommand{\subsectionautorefname}{Section}
  \renewcommand{\subsubsectionautorefname}{Section}
}

% pkg: url
\renewcommand{\UrlFont}{\footnotesize\tt}
% \renewcommand{\thefootnote}{\arabic{footnote}}

% our custom commands
\renewcommand{\ttdefault}{cmtt} % use computer modern for teletype

%%% draft mode / toggle commands %%%
\usepackage{etoolbox}
\newtoggle{draft}
\settoggle{draft}{true} % change toggle for draft or final versions

\iftoggle{draft}{
  % if 'draft' toggle is true
  \overfullrule=10pt                       % highlight overfull hboxes
}{
  % if 'draft' toggle is false
  \PassOptionsToPackage{final}{showlabels} % hide labels on figures, etc
}

% if you're including existing papers into your thesis, it helps to put
% content behind a toggle (or conditional) so you only have to maintain
% and keep consistency on one copy. see "introduction.tex".
\newtoggle{thesis}
\settoggle{thesis}{true}

\usepackage[inline]{showlabels}
\renewcommand{\showlabelfont}{\sffamily \color{blue}}
\renewcommand{\showlabelsetlabel}[1]{\efbox{\showlabelfont #1}}
%%%%%%%%%%%%%%%%%%%%%%%%%%%%%%%%%%%%%%%%%%%%%%%%%%%%%%%%%%%%%%%%%%%%%%%%

%%% front matter %%%%%%%%%%%%%%%%%%%%%%%%%%%%%%%%%%%%%%%%%%%%%%%%%%%%%%%
\begin{document}

% title should be all caps
\title{CODING.CARE: GUIDEBOOKS FOR INTERSECTIONAL AI}
% use your full name!
% https://cs.stanford.edu/~knuth/news19.html
% "Let's celebrate everybody's full names"
\author{Sarah Ciston}

% major should be all caps
\majorfield{CINEMATIC ARTS (MEDIA ARTS AND PRACTICE)}

% date should be May, August, or December (when degrees are conferred)
\submitdate{May 2024}

%%% preface %%%%%%%%%%%%%%%%%%%%%%%%%%%%%%%%%%%%%%%%%%%%%%%%%%%%%%%%%%%%
\begin{preface}
  % \prefacesection{Dedication}
  % \input{dedication.tex}

  \prefacesection{Acknowledgements}
  % \input{acknowledgements.tex}

  {
  \hypersetup{hidelinks} % color all links black in the preface
  \tableofcontents
  \listoftables
  \listoffigures
  }

  \prefacesection{Abstract}
  % \input{abstract.tex}
\end{preface}

%%% introduction %%%%%%%%%%%%%%%%%%%%%%%%%%%%%%%%%%%%%%%%%%%%%%%%%%%%%%%
\chapter{Introduction}
  \label{ch:introduction}

\graphicspath{}
% \input{introduction.tex}
$body$


%%% chapter %%%%%%%%%%%%%%%%%%%%%%%%%%%%%%%%%%%%%%%%%%%%%%%%%%%%%%%%%%%%

% if you're "stapling" together papers, it's easy to include your paper
% directory by way of symlinks, or copying the entire paper as a
% subdirectory.
%
% for example, if your paper directory looks like the following:
%
%   foobar/          - top level paper directory
%   foobar/fig/      - where all graphics and figures live
%   foobar/paper.bib - bibliography
%   foobar/paper.tex - monolithic .tex file for paper
%
% then you might use the folloiwng:
%
%   \graphicspath{foobar/fig}
%   \addbibresource{foobar/paper.bib}
%   \input{foobar/paper.tex}
%
% note that you'll have to modify the input file to make sure that the
% preamble (\documentclass, etc.) isn't included. to make your life
% easier, you could use some TeX conditionals to make it seamless.
%
% this requires some planning, but enables you to edit the individual
% paper and thesis chapter without tracking and porting changes between
% multiple directories and repositories:
%
% for example, at the beginning of foobar/paper.tex (before
% \documentclass):
%
%   \newif\ifdissertation
%   \dissertationtrue      % (or \dissertationfalse for the standalone)
%
%   \ifdissertation
%   \else
%   \documentclass...
%   \fi
%
%   \ifdissertation
%   \else
%   \begin{document}
%   \fi
%
%   [...paper content here...]
%
%   \ifdissertation
%   \else
%   \end{document}
%   \fi

%%% chapters: lorem ipsum %%%%%%%%%%%%%%%%%%%%%%%%%%%%%%%%%%%%%%%%%%%%%%

% The following text is to test ToC alignment:
% - of extremely long chapter, section, subsection, and
%   subsubsection titles
% - when chapter numbers are double digits

% \chapter{This is a very long title which will take up more than one line
% lorem ipsum dolor sit amet, consectetur adipiscing elit, sed do eiusmod
% tempor incididunt ut labore et dolore magna aliqua. Ultrices vitae
% auctor eu augue ut lectus arcu. Enim nunc faucibus a pellentesque sit
% amet porttitor eget. Consequat mauris nunc congue nisi vitae.}
%   \label{ch:long-title}

% \section{Ut enim ad minim veniam, quis nostrud exercitation ullamco
% laboris nisi ut aliquip ex ea commodo consequat}

% \subsection{Duis aute irure dolor in reprehenderit in voluptate velit
% esse cillum dolore eu fugiat nulla pariatur. Excepteur sint occaecat
% cupidatat non proident, sunt in culpa qui officia deserunt mollit anim
% id est laborum.}

% \subsubsection{Etiam erat velit scelerisque in dictum non. Sit amet
% justo donec enim diam. Amet justo donec enim diam. Metus vulputate eu
% scelerisque felis imperdiet proin. In nulla posuere sollicitudin aliquam
% ultrices. Turpis in eu mi bibendum.}

% \chapter{Lorem Ipsum}
% % \chapter{Lorem Ipsum}
% % \chapter{Lorem Ipsum}
% % \chapter{Lorem Ipsum}
% % \chapter{Lorem Ipsum}
% % \chapter{Lorem Ipsum}
% % \chapter{Lorem Ipsum}
% % \chapter{Lorem Ipsum}
% \chapter{Etiam erat velit scelerisque in dictum non. Sit amet
% justo donec enim diam. Amet justo donec enim diam. Metus vulputate eu
% scelerisque felis imperdiet proin. In nulla posuere sollicitudin aliquam
% ultrices. Turpis in eu mi bibendum.}

% \begin{table}
% \centering
% \begin{tabular}{lS}
% \toprule
% $x$      & \textbf{value} \\
% \midrule
% a        & 1.23           \\
% b        & 3.456          \\
% c        & 100.0002       \\
% d        & 12345.0        \\
% \bottomrule
% \end{tabular}
% \caption[In hendrerit gravida rutrum quisque non tellus orci ac. Iaculis
%         urna id volutpat lacus laoreet non curabitur gravida arcu. Mauris
%         ultrices eros in cursus turpis massa. Sed tempus urna et pharetra
%         pharetra massa massa. Eget sit amet tellus cras adipiscing enim eu
%         turpis egestas. Morbi blandit cursus risus at ultrices.]
%         {In hendrerit gravida rutrum quisque non tellus orci ac. Iaculis
%         urna id volutpat lacus laoreet non curabitur gravida arcu.
%         Mauris ultrices eros in cursus turpis massa. Sed tempus urna et
%         pharetra pharetra massa massa. Eget sit amet tellus cras
%         adipiscing enim eu turpis egestas. Morbi blandit cursus risus at
%         ultrices.}
% \label{tbl:example-2}
% \end{table}

%%% conclusions %%%%%%%%%%%%%%%%%%%%%%%%%%%%%%%%%%%%%%%%%%%%%%%%%%%%%%%%
% \chapter{Conclusions}
%   \label{ch:conclusions}

% \graphicspath{}
% \input{conclusions}

%%% bibliography %%%%%%%%%%%%%%%%%%%%%%%%%%%%%%%%%%%%%%%%%%%%%%%%%%%%%%%
%
%  \printbibliography in biblatex is great, but doesn't allow for the
%  greatest customization, so we'll use the package biblatex + biber
%  backend to meet some requirements:
%
%  * bibliography should be an un-numbered chapter, and still have a
%    pdfbookmark and a line in the table of contents
%
%  * bibliography contents should be singlespace, and optionally a smaller
%    font
%
%  * first line of this "chapter" should be in the same spot as the first
%    line of preface sections (e.g., acknowledgement)
%
%  * we use \raggedright so things like URLs and DOIs aren't stretched out.
%
\clearpage
\chapter*{Bibliography}
\addcontentsline{toc}{chapter}{Bibliography}

\begin{singlespace}
  % increase penalty such that we don't break entries over pages
  % source: https://tex.stackexchange.com/a/43275
  \patchcmd{\bibsetup}{\interlinepenalty=5000}{\interlinepenalty=10000}{}{}

  % reduce spacing between each bibentry
  \setlength\bibitemsep{0.9\baselineskip}

  % don't justify-align entries: this prevents stretching out each line
  \raggedright
  \printbibliography[
    heading = none
  ]
\end{singlespace}

\end{document}

%
% note that you'll have to modify the input file to make sure that the
% preamble (\documentclass, etc.) isn't included. to make your life
% easier, you could use some TeX conditionals to make it seamless.
%
% this requires some planning, but enables you to edit the individual
% paper and thesis chapter without tracking and porting changes between
% multiple directories and repositories:
%
% for example, at the beginning of foobar/paper.tex (before
% \documentclass):
%
%   \newif\ifdissertation
%   \dissertationtrue      % (or \dissertationfalse for the standalone)
%
%   \ifdissertation
%   \else
%   \documentclass...
%   \fi
%
%   \ifdissertation
%   \else
%   \begin{document}
%   \fi
%
%   [...paper content here...]
%
%   \ifdissertation
%   \else
%   \end{document}
%   \fi

%%% chapters: lorem ipsum %%%%%%%%%%%%%%%%%%%%%%%%%%%%%%%%%%%%%%%%%%%%%%

% The following text is to test ToC alignment:
% - of extremely long chapter, section, subsection, and
%   subsubsection titles
% - when chapter numbers are double digits

% \chapter{This is a very long title which will take up more than one line
% lorem ipsum dolor sit amet, consectetur adipiscing elit, sed do eiusmod
% tempor incididunt ut labore et dolore magna aliqua. Ultrices vitae
% auctor eu augue ut lectus arcu. Enim nunc faucibus a pellentesque sit
% amet porttitor eget. Consequat mauris nunc congue nisi vitae.}
%   \label{ch:long-title}

% \section{Ut enim ad minim veniam, quis nostrud exercitation ullamco
% laboris nisi ut aliquip ex ea commodo consequat}

% \subsection{Duis aute irure dolor in reprehenderit in voluptate velit
% esse cillum dolore eu fugiat nulla pariatur. Excepteur sint occaecat
% cupidatat non proident, sunt in culpa qui officia deserunt mollit anim
% id est laborum.}

% \subsubsection{Etiam erat velit scelerisque in dictum non. Sit amet
% justo donec enim diam. Amet justo donec enim diam. Metus vulputate eu
% scelerisque felis imperdiet proin. In nulla posuere sollicitudin aliquam
% ultrices. Turpis in eu mi bibendum.}

% \chapter{Lorem Ipsum}
% % \chapter{Lorem Ipsum}
% % \chapter{Lorem Ipsum}
% % \chapter{Lorem Ipsum}
% % \chapter{Lorem Ipsum}
% % \chapter{Lorem Ipsum}
% % \chapter{Lorem Ipsum}
% % \chapter{Lorem Ipsum}
% \chapter{Etiam erat velit scelerisque in dictum non. Sit amet
% justo donec enim diam. Amet justo donec enim diam. Metus vulputate eu
% scelerisque felis imperdiet proin. In nulla posuere sollicitudin aliquam
% ultrices. Turpis in eu mi bibendum.}

% \begin{table}
% \centering
% \begin{tabular}{lS}
% \toprule
% $x$      & \textbf{value} \\
% \midrule
% a        & 1.23           \\
% b        & 3.456          \\
% c        & 100.0002       \\
% d        & 12345.0        \\
% \bottomrule
% \end{tabular}
% \caption[In hendrerit gravida rutrum quisque non tellus orci ac. Iaculis
%         urna id volutpat lacus laoreet non curabitur gravida arcu. Mauris
%         ultrices eros in cursus turpis massa. Sed tempus urna et pharetra
%         pharetra massa massa. Eget sit amet tellus cras adipiscing enim eu
%         turpis egestas. Morbi blandit cursus risus at ultrices.]
%         {In hendrerit gravida rutrum quisque non tellus orci ac. Iaculis
%         urna id volutpat lacus laoreet non curabitur gravida arcu.
%         Mauris ultrices eros in cursus turpis massa. Sed tempus urna et
%         pharetra pharetra massa massa. Eget sit amet tellus cras
%         adipiscing enim eu turpis egestas. Morbi blandit cursus risus at
%         ultrices.}
% \label{tbl:example-2}
% \end{table}

%%% conclusions %%%%%%%%%%%%%%%%%%%%%%%%%%%%%%%%%%%%%%%%%%%%%%%%%%%%%%%%
% \chapter{Conclusions}
%   \label{ch:conclusions}

% \graphicspath{}
% \input{conclusions}

%%% bibliography %%%%%%%%%%%%%%%%%%%%%%%%%%%%%%%%%%%%%%%%%%%%%%%%%%%%%%%
%
%  \printbibliography in biblatex is great, but doesn't allow for the
%  greatest customization, so we'll use the package biblatex + biber
%  backend to meet some requirements:
%
%  * bibliography should be an un-numbered chapter, and still have a
%    pdfbookmark and a line in the table of contents
%
%  * bibliography contents should be singlespace, and optionally a smaller
%    font
%
%  * first line of this "chapter" should be in the same spot as the first
%    line of preface sections (e.g., acknowledgement)
%
%  * we use \raggedright so things like URLs and DOIs aren't stretched out.
%
\clearpage
\chapter*{Bibliography}
\addcontentsline{toc}{chapter}{Bibliography}

\begin{singlespace}
  % increase penalty such that we don't break entries over pages
  % source: https://tex.stackexchange.com/a/43275
  \patchcmd{\bibsetup}{\interlinepenalty=5000}{\interlinepenalty=10000}{}{}

  % reduce spacing between each bibentry
  \setlength\bibitemsep{0.9\baselineskip}

  % don't justify-align entries: this prevents stretching out each line
  \raggedright
  \printbibliography[
    heading = none
  ]
\end{singlespace}

\end{document}

%
% note that you'll have to modify the input file to make sure that the
% preamble (\documentclass, etc.) isn't included. to make your life
% easier, you could use some TeX conditionals to make it seamless.
%
% this requires some planning, but enables you to edit the individual
% paper and thesis chapter without tracking and porting changes between
% multiple directories and repositories:
%
% for example, at the beginning of foobar/paper.tex (before
% \documentclass):
%
%   \newif\ifdissertation
%   \dissertationtrue      % (or \dissertationfalse for the standalone)
%
%   \ifdissertation
%   \else
%   \documentclass...
%   \fi
%
%   \ifdissertation
%   \else
%   \begin{document}
%   \fi
%
%   [...paper content here...]
%
%   \ifdissertation
%   \else
%   \end{document}
%   \fi

%%% chapters: lorem ipsum %%%%%%%%%%%%%%%%%%%%%%%%%%%%%%%%%%%%%%%%%%%%%%

% The following text is to test ToC alignment:
% - of extremely long chapter, section, subsection, and
%   subsubsection titles
% - when chapter numbers are double digits

% \chapter{This is a very long title which will take up more than one line
% lorem ipsum dolor sit amet, consectetur adipiscing elit, sed do eiusmod
% tempor incididunt ut labore et dolore magna aliqua. Ultrices vitae
% auctor eu augue ut lectus arcu. Enim nunc faucibus a pellentesque sit
% amet porttitor eget. Consequat mauris nunc congue nisi vitae.}
%   \label{ch:long-title}

% \section{Ut enim ad minim veniam, quis nostrud exercitation ullamco
% laboris nisi ut aliquip ex ea commodo consequat}

% \subsection{Duis aute irure dolor in reprehenderit in voluptate velit
% esse cillum dolore eu fugiat nulla pariatur. Excepteur sint occaecat
% cupidatat non proident, sunt in culpa qui officia deserunt mollit anim
% id est laborum.}

% \subsubsection{Etiam erat velit scelerisque in dictum non. Sit amet
% justo donec enim diam. Amet justo donec enim diam. Metus vulputate eu
% scelerisque felis imperdiet proin. In nulla posuere sollicitudin aliquam
% ultrices. Turpis in eu mi bibendum.}

% \chapter{Lorem Ipsum}
% % \chapter{Lorem Ipsum}
% % \chapter{Lorem Ipsum}
% % \chapter{Lorem Ipsum}
% % \chapter{Lorem Ipsum}
% % \chapter{Lorem Ipsum}
% % \chapter{Lorem Ipsum}
% % \chapter{Lorem Ipsum}
% \chapter{Etiam erat velit scelerisque in dictum non. Sit amet
% justo donec enim diam. Amet justo donec enim diam. Metus vulputate eu
% scelerisque felis imperdiet proin. In nulla posuere sollicitudin aliquam
% ultrices. Turpis in eu mi bibendum.}

% \begin{table}
% \centering
% \begin{tabular}{lS}
% \toprule
% $x$      & \textbf{value} \\
% \midrule
% a        & 1.23           \\
% b        & 3.456          \\
% c        & 100.0002       \\
% d        & 12345.0        \\
% \bottomrule
% \end{tabular}
% \caption[In hendrerit gravida rutrum quisque non tellus orci ac. Iaculis
%         urna id volutpat lacus laoreet non curabitur gravida arcu. Mauris
%         ultrices eros in cursus turpis massa. Sed tempus urna et pharetra
%         pharetra massa massa. Eget sit amet tellus cras adipiscing enim eu
%         turpis egestas. Morbi blandit cursus risus at ultrices.]
%         {In hendrerit gravida rutrum quisque non tellus orci ac. Iaculis
%         urna id volutpat lacus laoreet non curabitur gravida arcu.
%         Mauris ultrices eros in cursus turpis massa. Sed tempus urna et
%         pharetra pharetra massa massa. Eget sit amet tellus cras
%         adipiscing enim eu turpis egestas. Morbi blandit cursus risus at
%         ultrices.}
% \label{tbl:example-2}
% \end{table}

%%% conclusions %%%%%%%%%%%%%%%%%%%%%%%%%%%%%%%%%%%%%%%%%%%%%%%%%%%%%%%%
% \chapter{Conclusions}
%   \label{ch:conclusions}

% \graphicspath{}
% \input{conclusions}

%%% bibliography %%%%%%%%%%%%%%%%%%%%%%%%%%%%%%%%%%%%%%%%%%%%%%%%%%%%%%%
%
%  \printbibliography in biblatex is great, but doesn't allow for the
%  greatest customization, so we'll use the package biblatex + biber
%  backend to meet some requirements:
%
%  * bibliography should be an un-numbered chapter, and still have a
%    pdfbookmark and a line in the table of contents
%
%  * bibliography contents should be singlespace, and optionally a smaller
%    font
%
%  * first line of this "chapter" should be in the same spot as the first
%    line of preface sections (e.g., acknowledgement)
%
%  * we use \raggedright so things like URLs and DOIs aren't stretched out.
%
\clearpage
\chapter*{Bibliography}
\addcontentsline{toc}{chapter}{Bibliography}

\begin{singlespace}
  % increase penalty such that we don't break entries over pages
  % source: https://tex.stackexchange.com/a/43275
  \patchcmd{\bibsetup}{\interlinepenalty=5000}{\interlinepenalty=10000}{}{}

  % reduce spacing between each bibentry
  \setlength\bibitemsep{0.9\baselineskip}

  % don't justify-align entries: this prevents stretching out each line
  \raggedright
  \printbibliography[
    heading = none
  ]
\end{singlespace}

\end{document}

%
% note that you'll have to modify the input file to make sure that the
% preamble (\documentclass, etc.) isn't included. to make your life
% easier, you could use some TeX conditionals to make it seamless.
%
% this requires some planning, but enables you to edit the individual
% paper and thesis chapter without tracking and porting changes between
% multiple directories and repositories:
%
% for example, at the beginning of foobar/paper.tex (before
% \documentclass):
%
%   \newif\ifdissertation
%   \dissertationtrue      % (or \dissertationfalse for the standalone)
%
%   \ifdissertation
%   \else
%   \documentclass...
%   \fi
%
%   \ifdissertation
%   \else
%   \begin{document}
%   \fi
%
%   [...paper content here...]
%
%   \ifdissertation
%   \else
%   \end{document}
%   \fi

%%% chapters: lorem ipsum %%%%%%%%%%%%%%%%%%%%%%%%%%%%%%%%%%%%%%%%%%%%%%

% The following text is to test ToC alignment:
% - of extremely long chapter, section, subsection, and
%   subsubsection titles
% - when chapter numbers are double digits

% \chapter{This is a very long title which will take up more than one line
% lorem ipsum dolor sit amet, consectetur adipiscing elit, sed do eiusmod
% tempor incididunt ut labore et dolore magna aliqua. Ultrices vitae
% auctor eu augue ut lectus arcu. Enim nunc faucibus a pellentesque sit
% amet porttitor eget. Consequat mauris nunc congue nisi vitae.}
%   \label{ch:long-title}

% \section{Ut enim ad minim veniam, quis nostrud exercitation ullamco
% laboris nisi ut aliquip ex ea commodo consequat}

% \subsection{Duis aute irure dolor in reprehenderit in voluptate velit
% esse cillum dolore eu fugiat nulla pariatur. Excepteur sint occaecat
% cupidatat non proident, sunt in culpa qui officia deserunt mollit anim
% id est laborum.}

% \subsubsection{Etiam erat velit scelerisque in dictum non. Sit amet
% justo donec enim diam. Amet justo donec enim diam. Metus vulputate eu
% scelerisque felis imperdiet proin. In nulla posuere sollicitudin aliquam
% ultrices. Turpis in eu mi bibendum.}

% \chapter{Lorem Ipsum}
% \chapter{Lorem Ipsum}
% \chapter{Lorem Ipsum}
% \chapter{Lorem Ipsum}
% \chapter{Lorem Ipsum}
% \chapter{Lorem Ipsum}
% \chapter{Lorem Ipsum}
% \chapter{Lorem Ipsum}
% \chapter{Etiam erat velit scelerisque in dictum non. Sit amet
% justo donec enim diam. Amet justo donec enim diam. Metus vulputate eu
% scelerisque felis imperdiet proin. In nulla posuere sollicitudin aliquam
% ultrices. Turpis in eu mi bibendum.}

% \begin{table}
% \centering
% \begin{tabular}{lS}
% \toprule
% $x$      & \textbf{value} \\
% \midrule
% a        & 1.23           \\
% b        & 3.456          \\
% c        & 100.0002       \\
% d        & 12345.0        \\
% \bottomrule
% \end{tabular}
% \caption[In hendrerit gravida rutrum quisque non tellus orci ac. Iaculis
%         urna id volutpat lacus laoreet non curabitur gravida arcu. Mauris
%         ultrices eros in cursus turpis massa. Sed tempus urna et pharetra
%         pharetra massa massa. Eget sit amet tellus cras adipiscing enim eu
%         turpis egestas. Morbi blandit cursus risus at ultrices.]
%         {In hendrerit gravida rutrum quisque non tellus orci ac. Iaculis
%         urna id volutpat lacus laoreet non curabitur gravida arcu.
%         Mauris ultrices eros in cursus turpis massa. Sed tempus urna et
%         pharetra pharetra massa massa. Eget sit amet tellus cras
%         adipiscing enim eu turpis egestas. Morbi blandit cursus risus at
%         ultrices.}
% \label{tbl:example-2}
% \end{table}

% %%% conclusions %%%%%%%%%%%%%%%%%%%%%%%%%%%%%%%%%%%%%%%%%%%%%%%%%%%%%%%%
% \chapter{Conclusions}
%   \label{ch:conclusions}

% \graphicspath{}
% \input{conclusions}

%%% bibliography %%%%%%%%%%%%%%%%%%%%%%%%%%%%%%%%%%%%%%%%%%%%%%%%%%%%%%%
%
%  \printbibliography in biblatex is great, but doesn't allow for the
%  greatest customization, so we'll use the package biblatex + biber
%  backend to meet some requirements:
%
%  * bibliography should be an un-numbered chapter, and still have a
%    pdfbookmark and a line in the table of contents
%
%  * bibliography contents should be singlespace, and optionally a smaller
%    font
%
%  * first line of this "chapter" should be in the same spot as the first
%    line of preface sections (e.g., acknowledgement)
%
%  * we use \raggedright so things like URLs and DOIs aren't stretched out.
%
% \clearpage
% \chapter*{Bibliography}
% \addcontentsline{toc}{chapter}{Bibliography}

% \begin{singlespace}
% %   % increase penalty such that we don't break entries over pages
% %   % source: https://tex.stackexchange.com/a/43275
%   \patchcmd{\bibsetup}{\interlinepenalty=5000}{\interlinepenalty=10000}{}{}

% %   % reduce spacing between each bibentry
%   \setlength\bibitemsep{0.9\baselineskip}

% %   % don't justify-align entries: this prevents stretching out each line
%   \raggedright
%   \printbibliography[
%     heading = none
%   ]
% \end{singlespace}

\end{document}
